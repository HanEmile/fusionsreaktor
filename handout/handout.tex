\documentclass[a4paper,12pt]{scrartcl}
\usepackage[top=1cm, bottom=1cm, left=1cm, right=1cm]{geometry}
\usepackage[utf8]{inputenc}
\usepackage[english,ngerman]{babel}
\usepackage{amsmath}
\usepackage{tabularx}
\usepackage[hidelinks]{hyperref}

\usepackage{graphicx}
\usepackage{caption}
\usepackage{lmodern}
\usepackage{textcomp}
\usepackage[onehalfspacing]{setspace}

\usepackage{tikz}
\usetikzlibrary{positioning}

\usepackage{listings}
\lstset{ %
  frame=single,	                   % adds a frame around the code
  language=Python,                 % the language of the code
  numbers=left,                    % where to put the line-numbers; possible values are (none, left, right)
  stepnumber=1,                    % the step between two line-numbers. If it's 1, each line will be numbered
}

\usepackage{hyperref}
\hypersetup{
colorlinks,
citecolor=black,
filecolor=black,
linkcolor=black,
urlcolor=black
}

\newcommand{\bold}{\textbf}
\renewcommand{\arraystretch}{1.5}
\newcommand{ \rarrow }{\( \rightarrow \)}

\setlength\parindent{0pt}
\setlength\parskip{10pt}
\setlength{\footskip}{30pt}
\setcounter{tocdepth}{3}

%%% MAIN DOCUMENT BEGINS HERE %%%

\begin{document}

% \title{ Fusionsreaktoren }
% \subtitle{Physik LK \the\year}

% \author{
%   Leon Schwarzer
%   \and
%   Emile Hansmaennel
% }
%
% \date{\today}
%
% \maketitle

\section*{Fusionsreaktoren Handout}

\paragraph{Was ist ein Fusionsreaktor?}

\begin{itemize}
  \item Ein \textbf{Kernfusionsreaktor} oder \textbf{Fusionsreaktor} ist eine
  technische Anlage, in der die \textbf{Kernfusion} von Deuterium \( {}^{2}De \)
   und Tritium \( {}^{3}Tr \) als Thermonukleare Reaktion abläuft.
\end{itemize}

\paragraph{Warum nutzen wir zurzeit keine Fusionsreaktoren zur Energiegewinnung?}

\begin{itemize}
  \item Es muss zu viel energie in den Reaktor gegeben werden um Energie wieder
  rauszubekommen. \( \rightarrow \) Die Energiebillanz ist negativ.
\end{itemize}

\paragraph{Was für Arten von Fusionsreaktoren gibt es?}

\subparagraph{Stellatoren (USA)}
\begin{itemize}
  \item Erzeugen die Verdrillung des Feldes \textbf{ Komplizierte Formung ihrer
  Magnetspulen }
\end{itemize}

\subparagraph{Tokamaks (Sowiet-Union)}
\begin{itemize}
  \item Erzeugen die Verdrillung des Feldes durch \textbf{ Induzieren eines
  elektrischen Stroms im Plasma }

\end{itemize}

\end{document}
