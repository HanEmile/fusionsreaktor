\documentclass[aspectratio=169]{beamer}
\usetheme{Singapore}

\title{Fusionsreaktoren (Konzepte)}
\subtitle{}
\author{Leon Schwarzer und Emile Hansmaennel}
\institute{Theodor-Fliedner-Gymnasium}
\date{\today}

\begin{document}
  \begin{frame}
    \titlepage
  \end{frame}

  \section{Allgemeines}
    \begin{frame}
      \frametitle{Was ist ein Fusionsreaktor?}
      Ein Kernfusionsreaktor oder Fusionsreaktor ist eine
      technische Anlage, in der die Kernfusion von Deuterium und Tritium als
      thermonukleare Reaktion kontrolliert abläuft.
    \end{frame}

    \begin{frame}
      \frametitle{Warum nutzen wir ihn nicht?}
      Zu viel Energie rein \( \rightarrow \) zu wenig energie raus
    \end{frame}

  \section{Physikalische Grundlagen}
    \begin{frame}
      \frametitle{Was passiert?}
      \begin{itemize}
        \item Kernfusion von Deuterium und Tritium als thermonukleare Reaktion
      \end{itemize}
    \end{frame}

    \begin{frame}
      \frametitle{Deuterium-Tritium-Reaktion}
      \begin{itemize}
        \item Atomkerne verschmelzen zu einem neuem Kern
        \item Energie wird freigesetzt
        \item Atomkerne kommen sich sehr nahe (2,5 Femtometer)
        \item Masse Vorher \( > \) Masse Nachher \( \rightarrow \) Differenz wird zu Energie

        \begin{equation}
          D + T \rightarrow {}^4 H + n + 17,6~MeV
        \end{equation}

      \end{itemize}
    \end{frame}

    \begin{frame}
      \frametitle{Fusion mit magnetischem Plasmaeinschluss}
      Lorem ipsum dolor sit amet, consectetur adipisicing elit, sed do eiusmod tempor incididunt ut labore et dolore magna aliqua. Ut enim ad minim veniam, quis nostrud exercitation ullamco laboris nisi ut aliquip ex ea commodo consequat. Duis aute irure dolor in reprehenderit in voluptate velit esse cillum dolore eu fugiat nulla pariatur. Excepteur sint occaecat cupidatat non proident, sunt in culpa qui officia deserunt mollit anim id est laborum.
    \end{frame}

  \section{Technik}

    \begin{frame}
      \frametitle{Plasmaaufheizung}
      lorem
    \end{frame}

    \begin{frame}
      \frametitle{Magnetfeld}
      lorem
    \end{frame}

  \section{Brennstoff}

    \begin{frame}
      \frametitle{Vorkommen und Beschaffung}
      Lorem ipsum dolor sit amet, consectetur adipisicing elit, sed do eiusmod tempor incididunt ut labore et dolore magna aliqua. Ut enim ad minim veniam, quis nostrud exercitation ullamco laboris nisi ut aliquip ex ea commodo consequat. Duis aute irure dolor in reprehenderit in voluptate velit esse cillum dolore eu fugiat nulla pariatur. Excepteur sint occaecat cupidatat non proident, sunt in culpa qui officia deserunt mollit anim id est laborum.
    \end{frame}

    \begin{frame}
      \frametitle{Tritiumbrüten und Neutronenvermehrung}
      Lorem ipsum dolor sit amet, consectetur adipisicing elit, sed do eiusmod tempor incididunt ut labore et dolore magna aliqua. Ut enim ad minim veniam, quis nostrud exercitation ullamco laboris nisi ut aliquip ex ea commodo consequat. Duis aute irure dolor in reprehenderit in voluptate velit esse cillum dolore eu fugiat nulla pariatur. Excepteur sint occaecat cupidatat non proident, sunt in culpa qui officia deserunt mollit anim id est laborum.
    \end{frame}

    \begin{frame}
      \frametitle{Brennstoffnachfüllung}
      Lorem ipsum dolor sit amet, consectetur adipisicing elit, sed do eiusmod tempor incididunt ut labore et dolore magna aliqua. Ut enim ad minim veniam, quis nostrud exercitation ullamco laboris nisi ut aliquip ex ea commodo consequat. Duis aute irure dolor in reprehenderit in voluptate velit esse cillum dolore eu fugiat nulla pariatur. Excepteur sint occaecat cupidatat non proident, sunt in culpa qui officia deserunt mollit anim id est laborum.
    \end{frame}

    \begin{frame}
      \frametitle{Entfernen von Helium und Verunreinigungen}
      Lorem ipsum dolor sit amet, consectetur adipisicing elit, sed do eiusmod tempor incididunt ut labore et dolore magna aliqua. Ut enim ad minim veniam, quis nostrud exercitation ullamco laboris nisi ut aliquip ex ea commodo consequat. Duis aute irure dolor in reprehenderit in voluptate velit esse cillum dolore eu fugiat nulla pariatur. Excepteur sint occaecat cupidatat non proident, sunt in culpa qui officia deserunt mollit anim id est laborum.
    \end{frame}

  \section{Bau eines Reaktors}

    \begin{frame}
      \frametitle{Materialien / Chemikalien}
      \begin{itemize}
        \item Lorem ipsum dolor sit amet, consectetur adipisicing elit
        \item Lorem ipsum dolor sit amet, consectetur adipisicing elit,
        \item Lorem ipsum dolor sit amet, consectetur adipisicing elit, \textit{Eine Änderung!}
        \item Lorem ipsum dolor sit amet, consectetur adipisicing elit,
      \end{itemize}
    \end{frame}

    \begin{frame}
      \frametitle{Aufbau}
    \end{frame}

  \section{Stand der Forschung}

  \begin{frame}
    \frametitle{Wo sind wir zurzeit???}
  \end{frame}

  \section{Quellen / Weiteres}

    \begin{frame}
      \frametitle{Quellen}
      \begin{itemize}
        \item https://de.wikipedia.org/wiki/Kernfusionsreaktor
        \item https://www.iter.org/proj/inafewlines
      \end{itemize}
    \end{frame}

\end{document}
