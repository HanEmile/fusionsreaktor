\documentclass[aspectratio=169]{beamer}
\usepackage[utf8]{inputenc}
\usepackage[T1]{fontenc}

\usetheme{Singapore}

\hypersetup{pdfstartview={Fit}}

\title{Fusionsreaktoren (Konzepte)}
\subtitle{}
\author{Leon Schwarzer und Emile Hansmaennel}
\institute{Theodor-Fliedner-Gymnasium}
\date{\today}

\begin{document}
  \begin{frame}
    \titlepage
  \end{frame}

  \section{Allgemeines}
    \begin{frame}
      \frametitle{Was ist ein Fusionsreaktor?}
      Ein Kernfusionsreaktor oder Fusionsreaktor ist eine
      technische Anlage, in der die Kernfusion von Deuterium und Tritium als
      thermonukleare Reaktion kontrolliert abläuft.
    \end{frame}

    \begin{frame}
      \frametitle{Warum nutzen wir ihn nicht?}
      \center
      Zu viel Energie rein \( \rightarrow \) zu wenig energie raus
    \end{frame}

  \section{Physikalische Grundlagen}
    \begin{frame}
      \frametitle{Was passiert?}
      \center
      Kernfusion von Deuterium und Tritium als thermonukleare Reaktion
    \end{frame}

    \begin{frame}
      \frametitle{Deuterium-Tritium-Reaktion}
      \begin{itemize}
        \item Atomkerne verschmelzen zu einem neuem Kern
        \pause
        \item Energie wird freigesetzt
        \pause
        \item Atomkerne kommen sich sehr nahe (2,5 Femtometer)
        \pause
        \item Masse Vorher \( > \) Masse Nachher \( \rightarrow \) Differenz wird zu Energie
        \pause
      \end{itemize}
      \bigskip
      \begin{equation}
        D + T \rightarrow {}^4 H + n + 17,6~MeV
      \end{equation}
      \begin{equation}
        E = mc^2
        \end{equation}
    \end{frame}

    \begin{frame}
      \frametitle{Fusion mit magnetischem Plasmaeinschluss}
      Lorem ipsum dolor sit amet, consectetur adipisicing elit, sed do eiusmod tempor incididunt ut labore et dolore magna aliqua. Ut enim ad minim veniam, quis nostrud exercitation ullamco laboris nisi ut aliquip ex ea commodo consequat. Duis aute irure dolor in reprehenderit in voluptate velit esse cillum dolore eu fugiat nulla pariatur. Excepteur sint occaecat cupidatat non proident, sunt in culpa qui officia deserunt mollit anim id est laborum.
    \end{frame}

  \section{Technik}

    \begin{frame}
      \frametitle{Plasmaaufheizung}
      \begin{itemize}
        \item Plasma wird auf über 100 Mio. C aufgeheitzt
        \pause
        \item (Deuteriumkerne haben bei 100 Mio. C eine mittlere Geschwindigkeit von etwa 1000 km/s)
      \end{itemize}
    \end{frame}

    \begin{frame}
      \frametitle{Plasmaaufheizung}
      \framesubtitle{Elektrisches Aufheizen}
      Das Plasma ist ein elektrischer Leiter und kann mittels eines induzierten elektrischen Stroms aufgeheizt werden. Dabei ist das Plasma die Sekundärspule eines Transformators. Allerdings steigt die Leitfähigkeit des Plasmas mit steigender Temperatur, so dass der elektrische Widerstand ab etwa 20–30 Millionen Grad bzw. 2 keV nicht mehr ausreicht, das Plasma stärker zu erhitzen.
    \end{frame}

    \begin{frame}
      \frametitle{Plasmaaufheizung}
      \framesubtitle{Neutralteilchen-Einschuss}
      Beim Einschießen schneller neutraler Atome in das Plasma (neutral beam injection, kurz NBI) wird die kinetische Energie dieser Atome – die im Plasma sofort ionisiert werden – durch Stöße auf das Plasma übertragen, wodurch sich dieses aufheizt.
    \end{frame}

    \begin{frame}
      \frametitle{Plasmaaufheizung}
      \framesubtitle{Elektromagnetische Wellen}
      Mikrowellen können die Ionen und Elektronen im Plasma auf ihren Resonanzfrequenzen (Umlauffrequenz in der Schraubenlinie, die das Teilchen im Magnetfeld beschreibt) anregen und somit Energie in das Plasma übertragen. Diese Methoden des Aufheizens werden Ion Cyclotron Resonance Heating (ICRH), Electron Cyclotron Resonance Heating (ECRH) und Lower Hybrid Resonance Heating (LHRH) genannt.
    \end{frame}

    \begin{frame}
      \frametitle{Plasmaaufheizung}
      \framesubtitle{Magnetische Kompression}
      Das Plasma kann wie ein Gas durch schnelles (adiabatisches) Zusammenpressen erwärmt werden. Ein zusätzlicher Vorteil dieser Methode ist, dass zugleich die Plasmadichte erhöht wird. Nur von Magnetspulen mit veränderbarer Stromstärke erzeugte Magnetfelder sind geeignet, das Plasma zusammen zu pressen; von supraleitenden Magnetspulen erzeugte Magnetfelder sind dafür nicht geeignet, weil ihre Stärke unveränderlich ist.
    \end{frame}

    \begin{frame}
      \frametitle{Magnetfeld}
      lorem
    \end{frame}

  \section{Brennstoff}

    \begin{frame}
      \frametitle{Vorkommen und Beschaffung}
      Lorem ipsum dolor sit amet, consectetur adipisicing elit, sed do eiusmod tempor incididunt ut labore et dolore magna aliqua. Ut enim ad minim veniam, quis nostrud exercitation ullamco laboris nisi ut aliquip ex ea commodo consequat. Duis aute irure dolor in reprehenderit in voluptate velit esse cillum dolore eu fugiat nulla pariatur. Excepteur sint occaecat cupidatat non proident, sunt in culpa qui officia deserunt mollit anim id est laborum.
    \end{frame}

    \begin{frame}
      \frametitle{Tritiumbrüten und Neutronenvermehrung}
      Lorem ipsum dolor sit amet, consectetur adipisicing elit, sed do eiusmod tempor incididunt ut labore et dolore magna aliqua. Ut enim ad minim veniam, quis nostrud exercitation ullamco laboris nisi ut aliquip ex ea commodo consequat. Duis aute irure dolor in reprehenderit in voluptate velit esse cillum dolore eu fugiat nulla pariatur. Excepteur sint occaecat cupidatat non proident, sunt in culpa qui officia deserunt mollit anim id est laborum.
    \end{frame}

    \begin{frame}
      \frametitle{Brennstoffnachfüllung}
      Lorem ipsum dolor sit amet, consectetur adipisicing elit, sed do eiusmod tempor incididunt ut labore et dolore magna aliqua. Ut enim ad minim veniam, quis nostrud exercitation ullamco laboris nisi ut aliquip ex ea commodo consequat. Duis aute irure dolor in reprehenderit in voluptate velit esse cillum dolore eu fugiat nulla pariatur. Excepteur sint occaecat cupidatat non proident, sunt in culpa qui officia deserunt mollit anim id est laborum.
    \end{frame}

    \begin{frame}
      \frametitle{Entfernen von Helium und Verunreinigungen}
      Lorem ipsum dolor sit amet, consectetur adipisicing elit, sed do eiusmod tempor incididunt ut labore et dolore magna aliqua. Ut enim ad minim veniam, quis nostrud exercitation ullamco laboris nisi ut aliquip ex ea commodo consequat. Duis aute irure dolor in reprehenderit in voluptate velit esse cillum dolore eu fugiat nulla pariatur. Excepteur sint occaecat cupidatat non proident, sunt in culpa qui officia deserunt mollit anim id est laborum.
    \end{frame}

  \section{Bau eines Reaktors}

    \begin{frame}
      \frametitle{Materialien / Chemikalien}
      \begin{itemize}
        \item Lorem ipsum dolor sit amet, consectetur adipisicing elit
        \item Lorem ipsum dolor sit amet, consectetur adipisicing elit,
        \item Lorem ipsum dolor sit amet, consectetur adipisicing elit,
        \item Lorem ipsum dolor sit amet, consectetur adipisicing elit,
      \end{itemize}
    \end{frame}

    \begin{frame}
      \frametitle{Aufbau}
    \end{frame}

  \section{Stand der Forschung}

  \begin{frame}
    \frametitle{Wo sind wir zurzeit???}
  \end{frame}

  \section{Quellen / Weiteres}

    \begin{frame}
      \frametitle{Quellen}
      \begin{itemize}
        \item https://de.wikipedia.org/wiki/Kernfusionsreaktor
        \item https://www.iter.org/proj/inafewlines
      \end{itemize}
    \end{frame}

\end{document}
